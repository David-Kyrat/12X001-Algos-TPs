\documentclass[french]{article}
\usepackage[margin=1in,a4paper]{geometry}
\usepackage[utf8]{inputenc}
\usepackage[T1]{fontenc}
\usepackage[autolanguage]{numprint}
\usepackage{hyperref}
\usepackage[fleqn]{amsmath}
\usepackage{enumitem,amssymb}
\usepackage{graphicx}
\usepackage{xcolor}
\usepackage{amsthm}
\usepackage{amsfonts}
\usepackage{pdfpages}
%\usepackage{pgfplots}
\usepackage{fancyhdr}
\usepackage{pdfpages}
\usepackage{lastpage}
\usepackage{cleveref}
\usepackage{ragged2e}
\usepackage{listings}
\usepackage{algorithm,algorithmicx,algpseudocode}
%
\definecolor{dkgreen}{rgb}{0,0.6,0}
\definecolor{gray}{rgb}{0.5,0.5,0.5}
\definecolor{mauve}{rgb}{0.58,0,0.82}
%
\lstset{frame=tb,
  language=Python,
  aboveskip=3mm,
  belowskip=3mm,
  label={lst:code_direct}
  showstringspaces=false,
  columns=flexible,
  basicstyle={\small\ttfamily},
  numbers=none,
  numberstyle=\tiny\color{gray},
  keywordstyle=\color{blue},
  commentstyle=\color{dkgreen},
  stringstyle=\color{mauve},
  breaklines=true,
  breakatwhitespace=true,
  tabsize=3
}
%
\pagestyle{fancy}
\setlength{\headheight}{73.96703pt}
\addtolength{\topmargin}{-49.96703pt}
\setlength{\footskip}{45.81593pt}
\setlength{\parindent}{0cm}
%
\hypersetup{colorlinks=true,
    linkcolor=blue,
    urlcolor=blue,
}
\urlstyle{same}
%
\newcommand{\unilogo}[1]{\includegraphics[scale=#1]{images/unige.png}}
%
%
\definecolor{light-gray}{gray}{0.95}
\newcommand{\code}[1]{$\mbox{\colorbox{light-gray}{\texttt{#1}}}$}
\newcommand{\quo}[1]{``{#1}''}
%
\newcommand{\R}{\mathbb{R}}
\newcommand{\N}{\mathbb{N}}
\newcommand{\es}{\varnothing}
%\DeclareMathOperator{\Ima}{Im}
\renewcommand{\and}{\wedge}
\newcommand{\ra}{\rightarrow}
\newcommand{\inv}[1]{{#1}^{-1}}
\newcommand{\br}[1]{\{{#1}\}}
\newcommand{\fa}{\forall}
\newcommand{\te}{\exists}
\newcommand{\dom}[1]{\mathcal{D}_{#1}}
%
\newcommand{\set}[2]{\{{#1}\ |\ {#2}\}}
\newcommand{\w}{\omega}
\newcommand{\s}{\Sigma}
%
\newcommand{\xor}{\oplus}
\newcommand{\nb}{05}
\newcommand{\xsol}{x^{\star}}
%
\DeclareMathSymbol{*}{\mathbin}{symbols}{"01}
\pdfcompresslevel=0
\pdfobjcompresslevel=0
%
%
\title{\vspace{-2cm}
   {\huge Université de Genève \\ - \\ Sciences Informatiques} \\
    \vspace{0.6cm}
    \unilogo{0.38} \\ 
    \vspace{1.1cm}
    {\huge Algorithmique - TP \nb}
    \vspace{0.1cm}
}
\author{Noah Munz (19-815-489)}
\date{Dec 2022}

%
\lhead{Université de Genève \\ Sciences Informatiques}
\rhead{Noah Munz \\ Algorithmique - TP\nb}
\cfoot{} \lfoot{\hspace{-1.8cm} \unilogo{0.06}}
\rfoot{Page \thepage \hspace{0.5mm} / \pageref{LastPage} \hspace{-1cm}}
%
%
\begin{document}
%
\maketitle
\vspace{0.5cm}
\tableofcontents
\thispagestyle{empty}
\clearpage
\setcounter{page}{1}
%
%
\begin{center}
	{\huge TP \nb}
\end{center}
\vspace{0.3cm}
%
\section{Taquin / Fifteen Puzzle (5 Points)}

\subsection{test}

\textsubscript{text}
\begin{algorithm}
	\floatname{algorithm}{Algorithm}
	\algrenewcommand\algorithmicrequire{\textbf{Input: }}
	\algrenewcommand\algorithmicensure{\textbf{Output: }}
	\caption{}
	\label{algo:}
	\begin{algorithmic}[1]
		\Require $input$
		\Ensure $output$
		
		\State \textbf{return} $state$
		\State lul
	\end{algorithmic}
\end{algorithm}


Algorithm~\ref{algo:}

\begin{figure}[htbp]
	\centering
	
	\caption{<caption>}
	\label{<label>}
\end{figure}


\subsection{Fonction de coût}
\vspace{0.4cm}


\begin{enumerate}[label=(\alph*)]
	\item \textit{Quel est le rôle de $h(x)$ dans la fonction de coût $\hat{c}(x) = g(x) + h(x)$ ?}
	\item \textit{A quel type de recherche correspond la fonction de coût}
	\begin{enumerate}[label=\arabic*)]
		\item \textit{$\hat{c}(x) = g(x) + h(x)$ ?}
		\item \textit{$\hat{c}(x) = h(x)$ ?}
	\end{enumerate}
	\item  \textit{En se basant sur les critères discutés en cours concernant les propriétés de $\hat{c}$
	vis-à-vis d'une fonction de coût idéale $c(x)$, montrez que si l'on fait une recherche Branch-and-Bound 
	avec la fonction de coût définie au point précédent et que l'on trouve une solution, cette solution est optimale.}\\
\end{enumerate}

\begin{enumerate}[label=(\alph*)]

	\item Avant de parler du rôle de la fonction $h(x)$, il faut d'abord expliquer pourquoi il est avantageux (dans ce contexte) de garder notre $\hat{c}(x_j)$ le plus constant possible pour tout $j$.\\
	
	Soit $T$, l'arbre des possibilités de racine $x_0$ (parfois appelé \textit{root}) du jeu (où une solution de notre problème est simplement une suite de edge de $T$)
	et soit $x_j$ un noeud de $T$, i.e. $x_j$ est un état de notre problème.
	
	On veut garder $\hat{c}(x) \leq \hat{c}(y)$ pour tout enfants $y$ de $x$, car pour une recherche directe 
	qui minimise le coût $\hat{c}(x)$, le relation (3.3) du cours donne que $c(x) \leq c(y)$. \\
	
	Or $\forall \text{ enfant } y \text{ de } x:\ c(x) \leq c(y)$ 
	implique que $\hat{c}(\xsol) = c(\xsol) \ge c(x_0) = c(\textit{optimum})$. ($\xsol$ est un noeud solution) \\
	Donc vu que le coût des noeuds dans la descendance d'un $x_0$ et jusqu'à \(\xsol\) ne peut pas diminuer, et que $\xsol$ 
	est garantie d'être une solution optimale, le mieux qu'on puisse faire (pour minimiser le coût du chemin total $x_0 \ldots x_i\ldots \xsol$) est de ne pas augmenter
	le coût des $x_{i\ >0}$ i.e. le garder constant.

	En résumé, on veut garder les $\hat{c}(x_{i+k})$ le plus proche de $c(x_0)$ et donc des autres $\hat{c}(x_i)$ possible.\\
	Ce qui ce traduit par \quo{garder le coût} constant car la relation (3.3) nous garanti que $c(x_0)$ est le plus petit coût possible. (i.e. le coût de la solution qui minimise le plus le coût total)\\
	(Ce qui est particulièrement dur quand $\hat{c}$ est une mauvaise approximation i.e. les $\hat{c}(x)$ sont loin des $c(x)$)

 $c(x_0) \approx \hat{c}(x_0) \approx \hat{c}(x_i) \approx \hat{c}(x_{i+k}) \approx \hat{c}(\xsol)$ serait une recherche parfaite
 (pour un $\hat{c}(x_0)$ proche de $c(x_0)$,\ \(k, i \in \N\) )\\

On peut donc maintenant constater que le rôle de $h$ est d'imposer que le coût des $x_{i+k}$ ne peut rester constant que si les $g(x_{i+k})$ diminuent.
C'est à dire que si $g$ est une \quo{mauvaise} borne/fonction ($g$ est une borne inférieur du nombre de coup qui reste à faire), cela va directement se voir sur le coût total $\hat{c}$ 
et les $\hat{c}(x_{i+k})$ (pour les plus hautes valeurs de $k$). 
En effet on a que: 
\begin{align*}
	\hat{c}(x_{i+k}) \leq \hat{c}(x_i) &\Longleftrightarrow g(x_{i+k}) + h(x_{i+k})\ \leq\ g(x_{i}) + h(x_{i}) \\
	%&\Longleftrightarrow g(x_{i+k}) + h(x_{i+k}) \leq g(x_{i}) + h(x_{i})\\
	&\Longleftrightarrow g(x_{i+k}) + h(x_{i}) + k\ \leq\ g(x_{i}) + h(x_i)\\
	&\Longleftrightarrow g(x_{i+k}) + k\ \leq\ g(x_{i})\\
	&\Longleftrightarrow k\ \leq\ g(x_{i}) - g(x_{i+k})\\
\end{align*}

Ce qui veut dire que pour garder le coût constant, il faut que entre $g(x_{i})$ et $g(x_{i+k})$ on se soit rapproché de la solution optimale $\xsol$ de $k$ coups.\\
Or comme on fait seulement un mouvement par étape, (i.e. $\max(g(x_i) - g(x_{i+1})) = 1 $)\\
cela implique qu'à chaque étape on se rapproche de $\xsol$ autant que possible, i.e. qu'on prenne le meilleur chemin / E-Node à chaque étape.\\

Ce qui nous donne la relation \quo{on est sur le bon chemin} ($\hat{c}(x_{i+k})$ constants) si et seulement si on le nombre de case mal placé diminue (i.e. $g(x_{i+k}) < g(x_i)$).\\

\item \textit{A quel type de recherche correspond la fonction de coût}
\begin{enumerate}[label=\arabic*)]
	\item \textit{$\hat{c}(x) = g(x) + h(x)$ ?}\\
	C'est une recherche "least cost" (comme défini en (3.4) dans le cours), on est dans le cas d'un jeu
	où l'espace à explorer est un graph dirigé qui décrit les différents mouvements possibles à partir d'une configuration $x_0$.

	On cherche donc une configuration gagnante $\xsol$ telle que la longueur du chemin \textit{path($x_0$, $\xsol$)} 
	i.e. la somme des coûts des noeuds du chemin $(x_0 \ldots x_i \ldots \xsol)$ soit minimale.\\
	
	Pour ce faire on définit donc $h(x)$ comme étant la profondeur de la configuration $x$ dans l'arbre de recherche, (\textit{height})
	(i.e. le nombre de coups déjà fait / coût accumulé jusqu'à $x$) et $g(x)$ comme une lower bound du nombre de coups restant à faire pour atteindre $\xsol$.
	
	Comme ça, on a une estimation du nombre de coups minimum qui compose \textit{path($x_0$, $\xsol$)}. (un coup == un edge de l'arbre)\\
	
	\item \textit{$\hat{c}(x) = h(x)$ ?}\\
	D'après la relation (3.4) du cours, encore une fois, on aurait à faire à une recherche\\ à \textit{coût uniforme}.
	\begin{quote}
		\quo{\textit{On peut bien sûr choisir $g(x) = 0$ pour tout x. On parle de recherche à coût uniforme. Le Branch \& Bound trouvera
		l'optimum recherché, mais l'efficacité de l'exploration sera sans doute faible.}}
	\end{quote}
	On remarque que dans ce cas la recherche \quo{least cost} nous donne en fait un BFS (breadth first search).\\
	car seul la profondeur détermine le coût donc on va explorer les noeuds sur la même largeur de $x$ en priorité.
	Elle est donc très peu efficace, niveau complexité en temps car il n'y aucun \quo{guide} sur le chemin à prendre.\\

	\item Les critères et propriétés de \(\hat{c}(x)\) ayant été longuement abordés au point 1) on se contentera de rappeler le théorème 2 du cours:
	\begin{quote}
		\quo{\textit{Soit une recherche Branche \& Bound à coût minimum (least cost) 
		et soit une fonction \(\hat{c}\) qui vérifie que \(\hat{c}(x) \leq c(x) \) pour tous les noeuds $x$, 
		et \(\hat{c}(\xsol) = c(\xsol) \) pour les noeuds réponse $\xsol$.}}
	\end{quote}
	Donc en se basant sur le fait que l'équation 1 de l'énoncé $\hat{c}(x) = g(x) + h(x)$ vérifie bien les critères du théorème 2,
	il ne nous reste plus qu'à prouver que \(h(x) \leq g(x) + h(x) \) (car \(g(x) + h(x) \leq c(x) \)) et que \(h(\xsol) = h(\xsol) + g(\xsol)\) (car \(g(\xsol) + h(\xsol) = c(\xsol) \)):
	\begin{align*}
		h(x) &\leq g(x) + h(x)\\
		&\Longleftrightarrow h(x) - h(x) \leq g(x)\\
		&\Longleftrightarrow 0 \leq g(x)
	\end{align*} 
	Or $g(x)$ est une lower bound du nombre de coups restant à faire pour atteindre $\xsol$ elle donc évidemment toujours plus grande à 0. Sauf pour $g(\xsol)$ qui vaut 0.
	
	\begin{align*}
		h(\xsol) =&\ h(\xsol) + g(\xsol)\\
		\Longleftrightarrow&\  h(\xsol) = h(\xsol) + 0\\
		\Longleftrightarrow&\ h(\xsol) = h(\xsol)
	\end{align*}
	Donc par le théorème 2, si on trouve une solution avec cette fonction de coût (\(\hat{c}(x) =  h(x)\)) elle est optimale.
\end{enumerate}
\end{enumerate}
%\newpage

\stepcounter{subsection}

\subsection{Efficacité de l'algorithme}
\begin{enumerate}
\item Optimalité de la solution. \\
\includegraphics[scale=0.5]{images/AlgoTP5-c-h.png}
$.\qquad \qquad \qquad $\textit{Figure 1. (légende sur graph)}\\
\includegraphics[scale=0.5]{images/AlgoTP5-both-c.png}
$.\qquad \qquad \qquad $ \textit{Figure 2. (légende sur graph)}\\

Soit $n$ le nombre de coup optimal i.e. $g(x_0)$ et $m$ la taille de la matrice du board. Ici $m$ est fixé cependant on pourrait très bien utiliser la même implémention pour le problème avec un board de taille $m \times m$. pour $m$ variable (\textit{la version $m \times m$, est testée dans le fichier exo1\_3.py}).\\
(On a bien que l'axe $x$ sur les 2 graphs représente le $n$ passé à la fonction \code{gen\_disorder()}.)\\

Comme on le voit sur la figure 2, l'efficacité en terme de \quo{quelle solution est la plus optimale?} est relativement la même pour les 2 fonctions de coûts. En effet, on voit bien que le nombre de coups à faire grandit parfaitment linéairement avec le désordre (i.e. distance à la solution qui est juste le board trié de 1 à 16). Les 2 courbes sont parfaitement superposées. 
Pour un move de plus on a bien au max 1 noeud en plus à parcourir pour les 2. \\
Par contre au niveau de la complexité en temps, les résultats sont nettements différents pour les 2 comme on va le voir au point précédent.

\item  Complexité en temps.

\includegraphics[scale=0.5]{images/AlgoTP5-runtime-c-h-g3.png}
$.\qquad \qquad \qquad $ \textit{Figure 3}\\

\includegraphics[scale=0.5]{images/AlgoTP5-runtime-c-h2.png}
$.\qquad \qquad \qquad $ \textit{Figure 4}\\

Comme on peut le voir, la complexité en temps de la fonction de coût de la figure 3 ($h(x)+g(x)$) est en $O(\hat{c}(x_0)^2)= O(\hat{c}(\xsol)^2) = O(n^2)$

Car on fait en général un boucle avec $n$ itérations et sur chaque itération on effectue au maximum $n$ \quo{moves} sur le board. (voir fonction \code{update\_misplaced\_compute\_cost} de l'exo1.)
et c'est le \quo{bout de code} le plus cher niveau complexité qu'on fait, i.e. la complexité du tout serait juste $O(n^2)+O(n^2)+ \cdots + O(n)+ \cdots + O(1) = O(n^2)$ \\

On voit que le runtime de l'autre fonction de coût (Figure 4) fit bien mieux à une courbe d'ordre $O(n^3)$ 
(le polynôme que l'on voit en légende à été calculé avec la méthode des moindres carrées, i.e. ce polynôme est celui qui minimise les carrées des distances entre les points et la courbe.
Montrant qu'on a bien du $O(n^3)$).

Il est important de noter que le 1er graph mesure le temps en \textbf{10$*$ns} (i.e. \( 10^{-8} s\).) et le 2e en \textbf{ms} (i.e. \( 10^{-3} s\).)\\
L'écart entre les 2 deux est donc \textit{massivement} plus grand que ce qu'on pourrait penser au 1er coup d'oeil, on a bien un facteur de $\times 10'000$ (i.e. \( 10^5 * 10^{-8} = 10^{-3}\ \)s).
\end{enumerate}

\stepcounter{section}

\section{Plus court chemin (Extra 1 Point)}

\subsection{Fonction de coût}

La fonction de coût qui remplie les critères nécessaires pour garantir l'optimisation d'une solution est
tout simplement la même que celle utilisée pour le problème du taquin.

En effet le théorème 2 du cours et le \quo{guide} sur comment choisir sa fonction de coût sont
tellements généraux qu'on pourrait les appliquer à beaucoup de problèmes.
La seule chose qui va changer ici, fondamentalement, c'est la fonction $g(x)$ qui a été choisie pour être simplement 
la distance définie par la norme $L_1$ (i.e. \(\Vert * \Vert_1\)) 
entre la position de $x$ et la destination $b$.

%
%
\end{document}